\chapter{Introduzione}
\label{Introduzione}
\thispagestyle{empty}

\begin{quotation}
{\footnotesize
\noindent{\emph{``Terence: Rotta a nord con circospezione \\
Bud: Ehi, gli ordini li do io qui!\\
Terence: Ok, comante\\
Bud: Rotta a nord\\
Terence: Soltanto?\\
Bud: Con circospezione!''}
}
\begin{flushright}
Chi Trova un Amico Trova un Tesoro
\end{flushright}
}
\end{quotation}
\vspace{0.5cm}

\noindent %L'introduzione deve essere atomica, quindi non deve contenere n\`e sottosezioni n\`e paragrafi n\`e altro. Il titolo, il sommario e l'introduzione devono sembrare delle scatole cinesi, nel senso che lette in quest'ordine devono progressivamente svelare informazioni sul contenuto per incatenare l'attenzione del lettore e indurlo a leggere l'opera fino in fondo. L'introduzione deve essere tripartita, non graficamente ma logicamente:

La polarizzazione \'e un utilissimo strumento per lo studio e l'analisi delle rete sociali sulle opinioni all'interno di differenti aree di ricerca. Generalmente la polarizzazione pu\'o essere tranquillamente applicata all'interno di contesti politici, sociali e culturali permettendo di comprendere al meglio quali siano le vere opinioni delle persone riguardo tali argomenti. Una generica definizione della polarizzazione \'e la seguente:
\begin{quote} 
Divisione in due gruppi fortemente contrastanti per una serie di opinioni o credenze.
\end{quote}
Questo processo di analisi pu\'o assumere diversi significati a seconda dello scenario studiato. 
\begin{itemize}
\item \textit{Polarizzazione Politica}: divergenza di opinione su estremi ideologici.
\item \textit{Polarizzazione Sociale}: differenza di opinione all'interno delle societ\'a che possono essere scaturite da disuguaglianze sociali ed economiche.

\end{itemize}

La polarizzazione pu\'o comportate diversi cambiamenti sullo scenario in questione, in quanto mette in luce come la formazione di due grandi gruppi non consenta una diffusione democratica delle opinioni. A tal proposito \'e interessante notare come la divisione in queste due grandi partizioni generi alcune problematiche quali:
\begin{itemize}
\item La frammentazione della rete stessa.
\item L'isolamento delle opinioni. 
\end{itemize} 

In conclusione potremmo definire la polarizzazione come un un processo sociale per cui il gli utenti che vi partecipiano vengono divisi in due grandi sottogruppi aventi visioni, punti di vista ed opinioni differenti del problema in qestione, con alcuni individui che rimangono neutrali tra i due grandi gruppi.

Il problema che si pu\'o facilmente evincere risulta essere la formazione di due comunit\'a isolate che non comunicano tra loro, ci\'o comporta un problema di isolamento delle opinioni, cio\'e un utente che appartiene a quel gruppo difficilmente potr\'a ricevere informazioni o aderire alle idee del gruppo opposto.
Otteniamo un problema che genera la formazione degli \textit{Echo-Chambers}. Si definiscono \textit{Echo-Chambers} come:
\begin{quote} 
Una situazione in cui le informazioni, le idee e le credenze vengono rinforzate e amplificate perch\'e espresse all'interno dello stesso ambiente, rimanendo isolato.
\end{quote}

Un'altro problema che pu\o formare una forte polarizzazione delle opinioni e delle informazioni sono i Filter Bubble ovvero:

\begin{quote} 
Uno stato di un isolamento intellettuale che pu\'o essere ottenuto a partire da risultati di ricerche su siti che registrano la storia del comportamento dell'utente. Questi siti sono in grado di utilizzare informazioni sull'utente per scegliere selettivamente tra tutte le risposte quelle che vorrà vedere l'utente stesso. L'effetto \'e di isolare l'utente da informazioni che sono in contrasto con il suo punto di vista, effettivamente isolandolo nella sua bolla culturale o ideologica.
\end{quote}

Come precedentemente anticipato la polarizzazione è uno strumento che pu\'o essere facilmente utilizzato per individuare tutte queste problamatiche all'interno dei moderni Social Network come Facebook e Twitter e molti altri. Questo perch\'e sempre pi\'u si stanno facendo largo nella vita di tutti i giorni e le problematiche relativi a contesti sociali, culturali e politici vengono sempre pi\'u affrontati all'interno di queste piattaforme, in cui gli utenti si sentono sempre pi\'u liberi di poter espimere le proprie opinioni.
Il problema \'e che non \'e sempre possibile uscire dalle Filter Bubble perch\'e gli stessi social network tendono a indirizzare l'utente a visualizzare informazioni che potrebbero interessarli senza fargli confrontare con opinioni divergenti. 
Alla luce di questo grande problema il calcolo di unta polarizzazione pu\'o consetire agli amministratori dei social network di individuare i topic pi\'u polarizzati e consentire una diffusione democratica delle informazioni.

L'obiettivo della mia tesi consiste nell'utilizzare la polarizzazione per poter individuare quegli argomenti fortemente polarizzati e comprendere come tali informazioni vengono prodotte all'interno della rete sociale.
Lo sviluppo di questo strumento \'e stato effettuato sfruttando due algoritmi, presentati nei seguenti paper:
\begin{itemize}
\item \textit{Measuring Political Polarization: Twitter shows the two sides of Venezuela}:
Studia la diffusione delle informazioni all'interno di un \textit{endorsement graph} collezionando i dati relativi alle elezioni in Venezuela all'interno del \textit{social network Twitter}. Viene effettuato uno studio della polarizzazione all'interno di un contesto politico sfruttando la diffusione delle informazioni, l'\textit{endorsement graph} viene costruito partendo da un nodo che scrive un Tweet esprimendo la propria opinioni, formando un nodo, mentre eventuali follower di quell'utente che retwettano tale notizia sono nuovi nodi all'interno del grafo con archi uscenti verso il nodo che hanno retwettato. In questo modo viene generato un grafo basato sul retweet.
Una volta generato il grafo vengono catalogati i nodi in due categorie:
\begin{itemize}
\item \textit{Elite}: l'utente che ha tweettato un'opinione.
\item \textit{Listener}: l'utente che ha retwettato il tweet di uno o pi\'u nodi \textit{Elite}
\end{itemize}
Partendo da queste categorie viene calcolata la polarizzazione sfruttando il grado di ogni nodo.(Per una pi\'u dettagliata spiegazione si rimanda al Capitolo??)

\item \textit{Reducing Controversy by Connecting Opposing Views}:
Identifica la polarizzazione sfruttando la struttura del grafo. Il grafo viene generato utilizzando la medesima tecninca precedentemente illustrata, cos\'i come il \textit{social network} di riferimento. La differenza principale \'e che non vengono catalogati i nodi in due gruppi in base al loro comportamento nel grafo. Adotta la tecninca dei \textit{Random Walk} sfruttando la probabilit\'a di retweet per ottenere il valore assoluto della polarizzazione per ogni nodo appartenente all' \textit{endorsment graph}.
\end{itemize}


\section{Inquadramento generale}
La prima parte contiene una frase che spiega l'area generale dove si svolge il lavoro; una che spiega la sottoarea pi\`u specifica dove si svolge il lavoro e la terza, che dovrebbe cominciare con le seguenti parole ``lo scopo della tesi \`e \dots'', illustra l'obbiettivo del lavoro. Poi vi devono essere una o due frasi che contengano una breve spiegazione di cosa e come \`e stato fatto, delle attivit\`a� sperimentali, dei risultati ottenuti con una valutazione e degli sviluppi futuri. La prima parte deve essere circa una facciata e mezza o due

\section{Breve descrizione del lavoro}
La seconda parte deve essere una esplosione della prima e deve quindi mostrare in maniera pi\`u esplicita l'area dove si svolge il lavoro, le fonti bibliografiche pi\`u importanti su cui si fonda il lavoro in maniera sintetica (una pagina) evidenziando i lavori in letteratura che presentano attinenza con il lavoro affrontato in modo da mostrare da dove e perch\'e \`e sorta la tematica di studio. Poi si mostrano esplicitamente le realizzazioni, le direttive future di ricerca, quali sono i problemi aperti e quali quelli affrontati e si ripete lo scopo della tesi. Questa parte deve essere piena (ma non grondante come la sezione due) di citazioni bibliografiche e deve essere lunga circa 4 facciate.

\section{Struttura della tesi}
La terza parte contiene la descrizione della struttura della tesi ed \`e organizzata nel modo seguente.
``La tesi \`e strutturata nel modo seguente.

Nella sezione due si mostra \dots

Nella sez. tre si illustra \dots

Nella sez. quattro si descrive \dots

Nelle conclusioni si riassumono gli scopi, le valutazioni di questi e le prospettive future \dots

Nell'appendice A si riporta \dots (Dopo ogni sezione o appendice ci vuole un punto).''

I titoli delle sezioni da 2 a M-1 sono indicativi, ma bisogna cercare di mantenere un significato equipollente nel caso si vogliano cambiare. Queste sezioni possono contenere eventuali sottosezioni.

%``Terence: "Mi fai un gelato anche a me? Lo vorrei di pistacchio" \\
%Bud: "Non ce l'ho il pistacchio. C'ho la vaniglia, cioccolato, fragola, limone e caff�"\\
%Terence: "Ah bene. Allora fammi un cono di vaniglia e di pistacchio"\\
%Bud: "No, non ce l'ho il pistacchio. C'ho la vaniglia, cioccolato, fragola, limone e caff�"\\
%Terence: "Ah, va bene. Allora vediamo un po', fammelo al cioccolato, tutto coperto di pistacchio"\\
%Bud: "Ehi, macch� sei sordo? Ti ho detto che il pistacchio non ce l'ho!"\\
%Terence: "Ok ok, non c'� bisogno che t'arrabbi, no? Insomma, di che ce l'hai?"\\
%Bud: "Ce l'ho di vaniglia, cioccolato, fragola, limone e caff�!"\\
%Terence: "Ah, ho capito. Allora fammene uno misto: mettici la fragola, il cioccolato, la vaniglia, il limone e il caff�. Charlie, mi raccomando il pistacchio, eh"''}
