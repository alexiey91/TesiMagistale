\chapter{Introduzione}
\label{Introduzione}
%\thispagestyle{empty}



%\noindent %L'introduzione deve essere atomica, quindi non deve contenere n\`e sottosezioni n\`e paragrafi n\`e altro. Il titolo, il sommario e l'introduzione devono sembrare delle scatole cinesi, nel senso che lette in quest'ordine devono progressivamente svelare informazioni sul contenuto per incatenare l'attenzione del lettore e indurlo a leggere l'opera fino in fondo. L'introduzione deve essere tripartita, non graficamente ma logicamente:

La polarizzazione \'e un utilissimo strumento per lo studio e l'analisi delle rete sociali sulle opinioni all'interno di differenti aree di ricerca. Generalmente la polarizzazione pu\'o essere tranquillamente applicata all'interno di contesti politici, sociali e culturali permettendo di comprendere al meglio quali siano le vere opinioni delle persone riguardo tali argomenti. Una generica definizione della polarizzazione \'e la seguente:
\begin{quote} 
\textit{Divisione in due gruppi fortemente contrastanti per una serie di opinioni o credenze.}
\end{quote}
Questo processo di analisi pu\'o assumere diversi significati a seconda dello scenario studiato. 
\begin{itemize}
\item \textit{Polarizzazione Politica}: divergenza di opinione su estremi ideologici.
\item \textit{Polarizzazione Sociale}: differenza di opinione all'interno delle societ\'a che possono essere scaturite da disuguaglianze sociali ed economiche.

\end{itemize}

La polarizzazione pu\'o comportate diversi cambiamenti sullo scenario in questione, in quanto mette in luce come la formazione di due grandi gruppi non consenta una diffusione democratica delle opinioni. A tal proposito \'e interessante notare come la divisione in queste due grandi partizioni generi alcune problematiche quali:
\begin{itemize}
\item La frammentazione della rete stessa.
\item L'isolamento delle opinioni. 
\end{itemize} 

In conclusione potremmo definire la polarizzazione come un un processo sociale per cui gli utenti che vi partecipiano vengono divisi in due grandi sottogruppi aventi visioni, punti di vista ed opinioni differenti del problema in questione, con alcuni individui che rimangono neutrali tra i due grandi gruppi.

Il problema che si pu\'o facilmente evincere risulta essere la formazione di due comunit\'a isolate che non comunicano tra loro, ci\'o comporta un problema di isolamento delle opinioni, cio\'e un utente che appartiene a quel gruppo difficilmente potr\'a ricevere informazioni o aderire alle idee del gruppo opposto.
Otteniamo un problema che genera la formazione degli \textit{Echo-Chambers}. Si definiscono \textit{Echo-Chambers} come:
\begin{quote} 
Una situazione in cui le informazioni, le idee e le credenze vengono rinforzate e amplificate perch\'e espresse all'interno dello stesso ambiente, rimanendo isolato.
\end{quote}

Un'altro problema che pu\'o formare una forte polarizzazione delle opinioni e delle informazioni sono i Filter Bubble ovvero:

\begin{quote} 
Uno stato di un isolamento intellettuale che pu\'o essere ottenuto a partire da risultati di ricerche su siti che registrano la storia del comportamento dell'utente. Questi siti sono in grado di utilizzare informazioni sull'utente per scegliere selettivamente tra tutte le risposte quelle che vorr\'a vedere l'utente stesso. L'effetto \'e di isolare l'utente da informazioni che sono in contrasto con il suo punto di vista, effettivamente isolandolo nella sua bolla culturale o ideologica.
\end{quote}

Come precedentemente anticipato la polarizzazione \'e uno strumento che pu\'o essere facilmente utilizzato per individuare tutte queste problamatiche all'interno dei moderni Social Network come Facebook e Twitter e molti altri. Questo perch\'e sempre pi\'u si stanno facendo largo nella vita di tutti i giorni e le problematiche relativi a contesti sociali, culturali e politici vengono sempre pi\'u affrontati all'interno di queste piattaforme, in cui gli utenti si sentono sempre pi\'u liberi di poter espimere le proprie opinioni.
Il problema \'e che non \'e sempre possibile uscire dalle Filter Bubble perch\'e gli stessi social network tendono a indirizzare l'utente a visualizzare informazioni che potrebbero interessarli senza fargli confrontare con opinioni divergenti. 
Alla luce di questo grande problema il calcolo di una polarizzazione pu\'o consetire agli amministratori dei social network di individuare i topic pi\'u polarizzati e consentire una diffusione democratica delle informazioni.

L'obiettivo della mia tesi consiste nell'utilizzare la polarizzazione per poter individuare quegli argomenti fortemente polarizzati e comprendere come tali informazioni vengono prodotte all'interno della rete sociale.
Lo sviluppo di questo strumento \'e stato effettuato sfruttando due algoritmi, presentati nei seguenti paper:
\begin{itemize}
\item \textit{Measuring Political Polarization: Twitter shows the two sides of Venezuela}:
Studia la diffusione delle informazioni all'interno di un \textit{endorsement graph} collezionando i dati relativi alle elezioni in Venezuela all'interno del \textit{social network Twitter}. Viene effettuato uno studio della polarizzazione all'interno di un contesto politico sfruttando la diffusione delle informazioni, l'\textit{endorsement graph} viene costruito partendo da un nodo che pubblica nella rete un Tweet esprimendo la propria opinione, formando un nodo, mentre eventuali follower di quell'utente che \textit{retwettano} tale notizia sono nuovi nodi all'interno del grafo con archi uscenti verso il nodo che hanno \textit{retwettato}. In questo modo viene generato un grafo basato sul \textit{retweet}.
Una volta generato il grafo vengono catalogati i nodi in due categorie:
\begin{itemize}
\item \textit{Elite}: l'utente che ha \textit{tweettato} un'opinione.
\item \textit{Listener}: l'utente che ha \textit{retwettato} il tweet di uno o pi\'u nodi \textit{Elite}.
\end{itemize}
Partendo da queste categorie viene calcolata la polarizzazione sfruttando il grado di ogni nodo.(Per una pi\'u dettagliata spiegazione si rimanda al Capitolo??)

\item \textit{Reducing Controversy by Connecting Opposing Views}:
Identifica la polarizzazione sfruttando la struttura del grafo. Il grafo viene generato utilizzando la medesima tecnica precedentemente illustrata, inoltre anche questo algoritmo \'e stato studiato sul social netwrok \textit{Twitter}. La differenza principale \'e che non vengono catalogati i nodi in due gruppi in base al loro comportamento nel grafo. Adotta la tecninca dei \textit{Random Walk} sfruttando la probabilit\'a di retweet per ottenere il valore della polarizzazione per ogni nodo appartenente all' \textit{endorsment graph}. Per la spiegazione relativa al calcolo del valore assoluto della polarizzazione attraverso la tecnina dei \textit{Random Walk} si rimanda al capitolo ??.
\end{itemize}

Prima di poter effettuare il calcolo vero e proprio della polarizzazione occorre effettuare una prima scrematura, nel dettaglio attraverso la \textit{Sentiment Analysis}. Questa particolare tecnica consente di partizionare il grafo in due gruppi che per semplicit\'a chiameremo \textbf{Rossi} e \textbf{Blu}, nel dettaglio viene analizzato il testo contenuto in un tweet o in un post (a seconda del social network adottato) catalogandolo per un gruppo piuttosto che un altro a seconda del contenuto e all'affinit\'a col topic in questione. 
Per meglio comprendere cosa viene effettuato presentiamo la definizione di \textit{Sentiment Analysis}:
\begin{quote}
L'Analisi del sentiment o Sentiment analysis (ma anche opinion mining) \'e la maniera a cui ci si riferisce all'uso dell'elaborazione del linguaggio naturale, analisi testuale e linguistica computazionale per identificare ed estrarre informazioni soggettive da diverse fonti. 
\end{quote}

In conclusione l'analisi semantica consente di poter catalogare le informazione in base alla lora vicininanza alle opinioni di un gruppo piuttosto che ad un'altra, ed eventualmente scartare quelle informazioni che non sono di alcun interesse per l'utente. Tale operazione \'e possibile soltanto se la macchina \'e stata precedentemente istruita sul topic in questione, infatti si definisce \textit{training set} l'insieme delle informazioni di riferimento che consentono alla macchina di poter distinguere le opinioni a seconda del loro contenuto.

Dopo aver effettuato questa separazione o catalogazione delle informazioni \'e possibile identificare quali utenti siano pi\'u o meno vicini ai due poli di un'opinione. Ricapitolando partendo da \textit{post} o \textit{topic} viene effettuata la \textit{Sentiment analysis}, viene costruito l'\textit{endorsement Graph} ed infine calcolata la \textbf{polarizzazione}. 
Per concludere \'e stato effettuato anche uno studio per poter consentire di predirre il valore della polarizzazione in un periodo futuro, in questo modo gli amministratore dei social network possono effettuare eventuali accorgimenti alla rete consentendo una democratica diffusione delle opinioni, senza creare \textit{Echo-Chambers} e \textit{Filter Bubble}.  
La predizione \'e stata realizzata attraverso tecniche di \textit{Forecasting} molto utilizzate in contesti economici, in quanto consentono attraverso delle serie numeriche di poter predirre il valore nell'istante temporale successivo. Sfruttando queste particolirit\'a \'e stato possibile predirre il valore della polarizzazione nell'istante temporale successivo, nel dettaglio le tecniche utilizzate sono tre:
\begin{itemize}
\item \textit{Double exponential smoothing}
\item \textit{Linear regression}
\item \textit{Average window}
\end{itemize}
Per i fondamenti teorici si rimanda al Capitolo???.

Terminiamo questa sezione presentando i casi studio utilizzato. Per lo sviluppo della mia tesi ho deciso di analizzare  due argomenti che presentano due contesti differenti della polarizzazione:
\begin{itemize}
\item \textbf{Elezioni Regionali in Sicilia nel 2017}: permettendo di analizzare la polarizzazione in un contesto politico.
\item \textbf{Biotestamento}: permettendo di analizzare la polarizzazione in un constesto sociale.
\end{itemize}
I dati relativi a questi due topic sono stati raccolti sul social network \textit{Twitter}, in un periodo temporale che andava dal 01/09/2017 al 20/12/2017, sfruttando diversi \textit{hashtags} nel dettaglio per ogni topic sono stati scelti i 5 hashtag pi\'u utilizzati dagli utenti per esprimere la loro opinioni su questi differenti argomenti. Una volta collezionati diversi dati \'e stato fatto quanto precedentemente illustrato. 
Per quanto rigurda le elezioni regionali in sicilia si \'e decisono di raccogliere i tweet relativi alle due grandi fazioni che hanno dominato la scena politica siciliana:
\begin{itemize}
\item  Il \textit{Movimento 5 Stelle}. 
\item \textit{Forza Italia}.
\end{itemize}
Innanzi tutto \'e stata una scelta dettata dai risultati conseguiti durante le suddette elezioni e dal fatto che in Italia non sono presenti soltanto due fazioni politiche come in molti altri paesi del mondo, quindi sarebbe risultato impossibile definire un valore polarizzato se avessimo considerato pi\'u di due fazioni politiche.
A tal proposito per quanto riguarda questo topic sono stati scartati i dati relativi ai candidati politici degli altri partiti politici e coalizione, utilizzando la \textit{Sentiment Analysis}.
I risultati ottenuti da questo topic hanno un comportamento interessante, cio\'e il cambiamento nel tempo della polarizazione, seguendo il trend riscontrato durante i sondaggi effettuati mensilmente. Nel dettaglio si pu\'o facilmente assistere ad un cambiamento di trend col passare del tempo, infatti in un primo istante c'\'e una totale polarizzazione verso il \textit{Movimento 5 stelle} per poi terminare con una polarizzazione tendente verso \textit{Forza Italia}.

Per quanto riguarda il \textit{Biotestamento} si \'e deciso di raccogliere i tweet relativi alla legge approvata il 14 Dicembre 2017 dal parlamento italiano, per analizzare la polarizzazioni in un contesto sociale.
La polarizzazione riguardava una valutazione positiva o negativa riguardo questa legge, infatti si \'e riscontrata una fortissima polarizzazione verso i contrari all'attuazione di tale legge. Ci\'o pu\'o essere facilmente addidato al contesto religioso e sociale presente in Italia, ovvero la forte presenza di una societ\'a cattolica che \'e contraria  all'attuazione del biotestamento. I risultati hanno confermato il contesto sociale cio\'e che la presenza relegiosa ha letteralmente dominato anche all'interno di \textit{Twitter} favorendo la creazione di un \textit{Echo-chambers} che non permette agli utenti di poter visualizzare opinioni discordanti rispetto alla loro opinioni.

In conclusione questi due Topic hanno contribuito a confermare quanto precedentemente spiegato all'interno di questo capitolo e cio\'e che la polarizzazione \'e un potentissimo strumento che consente di poter individuare le comunit\'a e consentire una possibile risoluzione degli \textit{Echo-chambers}. Eventuali sviluppo futuri relativi a questa tesi possono essere la possibilit\'a di eliminare la creazione degli \textit{Echo-chambers} utilizzando algoritmi per eliminare la controversia delle informazioni all'interno dei social network favorendo una facile diffusione delle informazioni. 
\begin{comment}
\section{Inquadramento generale}
La prima parte contiene una frase che spiega l'area generale dove si svolge il lavoro; una che spiega la sottoarea pi\`u specifica dove si svolge il lavoro e la terza, che dovrebbe cominciare con le seguenti parole ``lo scopo della tesi \`e \dots'', illustra l'obbiettivo del lavoro. Poi vi devono essere una o due frasi che contengano una breve spiegazione di cosa e come \`e stato fatto, delle attivit\`a� sperimentali, dei risultati ottenuti con una valutazione e degli sviluppi futuri. La prima parte deve essere circa una facciata e mezza o due

\section{Breve descrizione del lavoro}
La seconda parte deve essere una esplosione della prima e deve quindi mostrare in maniera pi\`u esplicita l'area dove si svolge il lavoro, le fonti bibliografiche pi\`u importanti su cui si fonda il lavoro in maniera sintetica (una pagina) evidenziando i lavori in letteratura che presentano attinenza con il lavoro affrontato in modo da mostrare da dove e perch\'e \`e sorta la tematica di studio. Poi si mostrano esplicitamente le realizzazioni, le direttive future di ricerca, quali sono i problemi aperti e quali quelli affrontati e si ripete lo scopo della tesi. Questa parte deve essere piena (ma non grondante come la sezione due) di citazioni bibliografiche e deve essere lunga circa 4 facciate.

\section{Struttura della tesi}
La terza parte contiene la descrizione della struttura della tesi ed \`e organizzata nel modo seguente.
``La tesi \`e strutturata nel modo seguente.

Nella sezione due si mostra \dots

Nella sez. tre si illustra \dots

Nella sez. quattro si descrive \dots

Nelle conclusioni si riassumono gli scopi, le valutazioni di questi e le prospettive future \dots

Nell'appendice A si riporta \dots (Dopo ogni sezione o appendice ci vuole un punto).''

I titoli delle sezioni da 2 a M-1 sono indicativi, ma bisogna cercare di mantenere un significato equipollente nel caso si vogliano cambiare. Queste sezioni possono contenere eventuali sottosezioni.
\end{comment}
%``Terence: "Mi fai un gelato anche a me? Lo vorrei di pistacchio" \\
%Bud: "Non ce l'ho il pistacchio. C'ho la vaniglia, cioccolato, fragola, limone e caff�"\\
%Terence: "Ah bene. Allora fammi un cono di vaniglia e di pistacchio"\\
%Bud: "No, non ce l'ho il pistacchio. C'ho la vaniglia, cioccolato, fragola, limone e caff�"\\
%Terence: "Ah, va bene. Allora vediamo un po', fammelo al cioccolato, tutto coperto di pistacchio"\\
%Bud: "Ehi, macch� sei sordo? Ti ho detto che il pistacchio non ce l'ho!"\\
%Terence: "Ok ok, non c'� bisogno che t'arrabbi, no? Insomma, di che ce l'hai?"\\
%Bud: "Ce l'ho di vaniglia, cioccolato, fragola, limone e caff�!"\\
%Terence: "Ah, ho capito. Allora fammene uno misto: mettici la fragola, il cioccolato, la vaniglia, il limone e il caff�. Charlie, mi raccomando il pistacchio, eh"''}
