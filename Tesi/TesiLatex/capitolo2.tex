\chapter{Stato dell'arte}
\label{capitolo2}
 

%\noindent Nella seconda sezione si riporta lo stato dell'arte del settore, un inquadramento dell'area di ricerca orientato a portare il lettore all'interno della problematica affrontata. Bisogna dimostrare di conoscere le cose fatte fino ad ora in questo campo e il perch\'e si sia reso necessario lo svolgimento di questo lavoro. Questa sezione deve essere grondante di citazioni bibliografiche~\cite{marco02bud}.

\section{Stato dell'arte}
All'interno delle reti sociali sta sempre pi\'u prendendo piede il problema della polarizzazione delle opinioni. Nel liguaggio comune il confronto tra individui ha sempre generato una forte controversia nelle opinioni oppure una situzione di neutralit\'a nelle opinione oppure una visione comune nelle opinione. 
I social netwrok hanno permesso all'utente di poter diffondere attraverso post, messaggi o espressioni audio video le proprie opinioni e\/o pensieri all'interno di una comunit\'a sociale. A tal proposito per favorire la diffusione delle diverse correnti di pensiero i social network stanno sempre pi\'u sviluppando algoritmi per permettere di identificare le comunità isolate che condividono un unico punto di vista di un problema.
La polarizzazione è un algoritmo matematico che applicato all'interno delle rete sociali permette di capire quanto un utente che accede per la prima volta all'interno di una rete sociale venga influenzato dagli altri utenti e quanto una news o un giudizio si propaga all'interno di una rete sociale.
Prima di poter illustrare questo algoritmo con le relative problematiche verr\'a illustrata una definizione di rete sociale.

\paragraph{Rete Sociale}
Una rete sociale consiste in un qualsiasi gruppo di individui connessi tra loro da diversi legami sociali. Per gli esseri umani i legami vanno dalla conoscenza casuale, ai rapporti di lavoro, ai vincoli familiari. Le reti sociali sono spesso usate come base di studi interculturali in sociologia, in antropologia, in etologia.

L'analisi delle reti sociali, ovvero la mappatura e la misurazione delle reti sociali, pu\'o essere condotta con un formalismo matematico usando la teoria dei grafi. In generale, il corpus teorico ed i modelli usati per lo studio delle reti sociali sono compresi nella cosiddetta social network analysis.

La ricerca condotta nell'ambito di diversi approcci disciplinari ha evidenziato come le reti sociali operino a pi\'u livelli e svolgano un ruolo cruciale nel determinare le modalit\'a di risoluzione di problemi e i sistemi di gestione delle organizzazioni, nonch\'e le possibilit\'a dei singoli individui di raggiungere i propri obiettivi.

\subparagraph{Le reti}
La diffusione del web e del termine social network ha creato negli ultimi anni alcune ambiguit\'a di significato. La rete sociale \'e infatti storicamente, in primo luogo, una rete fisica.

Rete sociale\'e, ad esempio, una comunità di lavoratori, che si incontra nei relativi circoli dopolavoristici e che costituisce una delle associazioni di promozione sociale. Esempi di reti sociali sono inoltre le comunit\'a di sportivi, attivi o sostenitori di eventi, le comunit\'a unite da problematiche strettamente lavorative e di tutela sindacale del diritto nel lavoro, le confraternite e in generale le comunit\'a basate sulla pratica comune di una religione e il ritrovo in chiese, templi, moschee, sinagoghe e altri luoghi di culto.

Una rete sociale si pu\'o inoltre basare su di un comune approccio educativo come nello scautismo, o nel pionierismo, di visione sociale, come nelle reti segrete della carboneria e della massoneria.

