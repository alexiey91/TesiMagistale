\chapter{Sviluppi futuri e conclusioni}
\label{capitolo7}
\thispagestyle{empty}


%\noindent Si mostrano le prospettive future di ricerca nell'area dove si \`e svolto il lavoro. Talvolta questa sezione pu\`o essere l'ultima sottosezione della precedente. Nelle conclusioni si deve richiamare l'area, lo scopo della tesi, cosa \`e stato fatto,come si valuta quello che si \`e fatto e si enfatizzano le prospettive future per mostrare come andare avanti nell'area di studio.
La polarizzazione è uno strumento che può essere utilizzato all'interno dei social media per poter analizzare alcuni comportamenti della rete a seconda del topic analizzato. Nel dettaglio attraverso i test è stata verificata la presenza di numerose comunità di utenti aventi una forte polarizzazione verso un unica visione. Attraverso questo strumento è possibile identificare questo tipo di comportamento perché in caso di reiterazione nel tempo può portare alla formazione di \textit{Echo-Chambers}. Una problematica che sta sempre più coinvolgendo gli amministratori dei social media.
Ci sono diversi meccanismi per poter impedire la formazione di questi gruppi isolati. Uno di questi è quello di cercare di abbassare il livello di controversia del grafo nel tempo. Il modo per farlo è quello di individuare i due gruppi fortemente polarizzati e cercare di farli comunicare tra di loro attraverso un congiungimento, cioè un arco.
Una soluzione sarebbe quella di suggerire ad utente che si trova all'interno di una rete fortemente polarizzata, di aggiungere, condividere o anche semplicemente mostrare dei post aventi una polarizzazione opposta a quella in cui ci si trova. 
Ovviamente non è un approccio vincente al 100\% perché molto spesso dipende dalla volontà dell'utente di incuriosirsi delle opinioni avverse. Però attraverso la polarizzazione si è potuto notare come molto spesso fossero presenti nodi grigi che risultavano imparziali verso i topic studiati, oppure come ci fossero anche nodi che non erano completamente polarizzati.

Un possibile impiego della polarizzazione potrebbe essere quello delle indagini di mercato all'interno dei social media, basterebbe che un'azienda lanciasse un \textit{hashtag} (nel caso di \textit{Twitter}), e studiasse che impatto abbia sulla rete. Le informazioni ottenute potrebbero facilmente essere utilizzate per intuire l'interesse del mercato di un nuovo prodotto. 

Come dimostrato attraverso lo studio ed i test effettuati attraverso le \textit{elezioni regionali in Sicilia}, è possibile scoprire la presa che ha un partito sulla rete piuttosto che un altro, in questo modo sarebbe fattibile fare delle previsioni sul risultato.
La previsione della polarizzazione potrebbe essere utilizzata in tutti i precedente casi illustrati all'interno di questa sezione proprio perché consente a chiunque di poter in parte conoscere quanto un certo argomento sarà oggetto di dibattito, e quindi capirne la presa sugli utenti.

In conclusione la polarizzazione attraverso un processo di analisi in profondità come la \textit{Sentiment Analysis} consente di comprendere al meglio la contrapposizione di idee in una rete sociale indipendentemente dal contesto di studio, sia che sia politico, sociale, etico ed economico.
Raffinare lo studio sui social media mediante algoritmi che non analizzino solo il testo dei messaggi pubblicati, ma che analizzi:
\begin{itemize}
\item immagini 
\item media vari
\item link a siti internet
\end{itemize} 
consentirebbe uno studio più preciso sulla polarizzazione e magari riuscirebbe anche a risolvere il problema dell'ironia che una tecnica come la sentiment analysis non sempre è in grado di classificare nella maniera più corretta.